\title{\textbf{CSCI 203 Place-out Project} \\
\texttt{milestone.pdf}
    \vspace{-6mm}}
\documentclass[12pt,letterpaper]{article}

\usepackage[utf8]{inputenc}
\usepackage [english]{babel}
\usepackage [autostyle, english = american]{csquotes}
\MakeOuterQuote{"}

\usepackage[margin=1in]{geometry}
\usepackage{setspace}
\setlength{\parindent}{0.2in}
\setlength{\parskip}{6pt}

%% To edit header %%
\usepackage{fancyhdr}
\pagestyle{fancy}
\lhead{Yash Mittal}

%% To insert code (set format) %%
\usepackage{listings}
\lstset{
  aboveskip=2mm,
  belowskip=2mm,
  showstringspaces=false,
  columns=flexible,
  basicstyle={\small\ttfamily},
  numbers=none,
  breaklines=true,
  breakatwhitespace=true,
  tabsize=4
}

%% To reduce Top Margin to 1 inch only %%
\usepackage{etoolbox}
\makeatletter
\patchcmd{\@maketitle}
  {\null\vskip 2em\begin{center}}
  {\centering}
  {}
  {}
\patchcmd{\@maketitle}
  {\end{center}}
  {}
  {}
  {}
\makeatother

\begin{document}

\date{}
\maketitle
\vspace{-12mm}

\setstretch{1.0}
\thispagestyle{fancy}

The program first accepts a \texttt{URL} and creates a \texttt{List} of all words appearing on the web page. A helper function \texttt{cleanContent()} is called to remove all instances of punctuation marks, digits and stop-words (stored in \texttt{stop-words.txt}). Next, abstracting functionality of a word stemmer, a \texttt{ModifiedStemmer Object} is created. The program then invokes functions to create and sort according to \texttt{frequency} a \texttt{List} of tuples \texttt{(word, frequency)}. Following is a sample program output with \texttt{MAX\textunderscore WORDS} set to \texttt{5}:

\begin{lstlisting}
    Please enter a URL http://www.eg.bucknell.edu/~csci203/placement/2016-spring/project/page1.html
    
    Here is the dictionary of words on that page: 
     {`website': 1, `love': 1, `spam': 10, `text': 1, `university': 1, `number': 1, `page': 2, `bucknell': 1, `example': 1, `cloud': 1}
    
    Here is the text cloud for your web page: 
    spam (10) 
    page (2) 
    website (1) 
    university (1) 
    text (1) 
\end{lstlisting}

For the first part of \texttt{CSCI 203} programming project, I ensured that the program is robust. At first glance, coding \texttt{stemContent()} appeared as challenging, so I read a renowned paper on a stemming algorithm formulated by Martin Porter. Thereafter, I implemented the Porter Stemmer \texttt{Class} in a unique way, using only one-fourth the number of lines but accounting for almost all major suffixes.

To improve readability, I encoded the above class in \texttt{modifiedStemmer.py} file. On the other hand, the main program majorly follows functional programming paradigm.

\section*{List of Functions}
\vspace{-2mm}
\begin{itemize}
    \item \texttt{getContent(url)} \\
        Returns as a \texttt{List} of splitted words on the user-entered website
    \item \texttt{cleanContent(wordList, stopwords)} \\
        Removes punctuation marks (\texttt{string.punctuation}) and digits using \texttt{string.replace} method, checks (and deletes) stopwords, returns as a \texttt{List} cleaned words
    \item \texttt{stemContent(wordList)} \\
        Stems words (usually, suffixed words) into root word, returns as a \texttt{List} stemmed words
    \item \texttt{filterContent(wordList)} \\
        Invokes helper functions to "clean" and "stem" words, returns as a \texttt{List} of pure words (no punctuations, stop-words or stemmed words)
    \item \texttt{findFrequency(finalWords)} \\
        Returns as a \texttt{Dict} of word-frequency pairs
    \item \texttt{findMostFrequentList(freqDict)} \\
        Finds defined number of most frequent words using \texttt{List.sort()}, adds word-frequency pairs to \texttt{freqStr} in required format, returns as a \texttt{List} tuples \texttt{(word, frequency)} in \texttt{DESC} order of frequency
    \item \texttt{ModifiedStemmer.update(self)} \\
        Updates local copy of word by removing already determined suffixes
    \item \texttt{ModifiedStemmer.isRestorable(self)} \\
        Returns \texttt{True} if `e' is to be restored (e.g., structur would return \texttt{True})
    \item \texttt{ModifiedStemmer.stemMajor(self)} \\
        Removes plurals and -ed or -ing or -er suffix
    \item \texttt{ModifiedStemmer.stemOther(self)} \\
        Removes major suffixes not accounted for by \texttt{stemMajor()}
    \item \texttt{ModifiedStemmer.stem(self, word)} \\
        Calls \texttt{stemMajor()} and \texttt{stemOther()} functions and returns the stemmed word to main program
\end{itemize}

\end{document}